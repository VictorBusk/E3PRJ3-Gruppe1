\section{Test}

For at sikre et stabilt og funktionelt produkt er der blevet gennemført tests af systemet. Systemet er løbende, gennem hele udviklings processen, blevet modultestet og integrationstestet, det vil sige at når der løbende er blevet implementeret dele på projektet er de blevet testet til den ønskede funktionalitet og præcision er opnået. Som afslutningstest af hele systemet, blev den tilhørende udspecificerede accepttest gennemført. Se accepttest under kravspecifikation i bilags sektionen. 

\subsubsection{Test forløb}
Gennem hele forløbet er der blevet testet for at sikre et acceptabelt produkt. Testene er løbene blevet gennemført i et samarbejde, mellem gruppens medlemmer for at kunne få alle modulerne til at kommunikere mellem hinanden. Produktet blev bygget systematisk op, så hele prototypens konstruktion blev bygget først samt. implementeringen af X Y modulet. Efter X og Y positioneringen og kommunikationen til PSoC-Master og DevKit var testet og oppe og køre, blev samme procedure udført for Z modulet og Sensor modulet. Ved test af sensor modul skulle de forskellige sensorer også testes og justeres, b.la. skulle bevægelsessensoren(P.I.R) testes og justeres så den var tilpasset prototypens størrelse. Dette samme gjorde sig gældende for afstandssensoren, for at sikre en sikker bevægelses zone for lampens hejse funktion.

Til test af XY og Z positionering blev Capsense slider-funktionen implementeret på de to PSoCs. Dette gjorde det muligt at manuelt teste kørslen af motorene i forskellige retninger samt at teste hele prototypens konstruktion. På denne måde kunne det observeres på hvad der fungerede og hvad der ikke gjorde og få ombygget og rekonstrueret eventuelt dårlige konstruktioner.

Til test af kommunikationen mellem Devkit8000 og modulerne, samt at kunne have et overblik over kommandoer i kø, blev der implementeret et Nokia5110 display på PSoc-Masteren. Dette display gjorde det muligt at se hvis kommunikationen fejlede mellem Devkit8000 og PSoC-Masteren og visse køens indhold.

Til test af hele software delen er der blevet brugt Tera Term, som gjorde det muligt at visse hvor eventuelle fejl var opstået i programmet. %Her skal indsættes nogle referencer.

\subsubsection{Accepttest}
Accepttesten er blevet udført efter modultest og integrationstest. Accepttesten er opbygget så den kan udføres af en hvilken som helst person. Gennemførelsen af accepttesten, dokumenteret i kravspecifikationen under bilag, er blevet udført af gruppen. Dette blev gjort ved at køre testen igennem punkt for punkt og udføre de gennemførte introduktioner der er beskrevet og der efter noterer resultatet i de her til tilhørende rubrikker. Testen er bygget op så denne gennemgår samtlige usecases der er, det vil sige at alle tænkelige funktioner i systemet bliver testet.

Til fejlfinding eller andet vil accepttesten til hver en tid kunne gengennemføres. Hvis dette ønskes henvises der til kravspecifikationen under bilag.   
