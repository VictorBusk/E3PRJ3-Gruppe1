\section{Metode}

Til beskrivelse af software er der brugt UML og 4+1 modellen.\footcite{4+1} Til de grafiske beskrivelser af systemet, både opbygningsmæssigt og funktionsmæssigt, er der blevet brugt SysML\footcite{SysML}.

UML står for Unified Modelling Language, diagramtyperne i UML kan deles op i to grupper: strukturdiagrammer og adfærdsdiagrammer. Det er ved hjælp af diagrammer at alle klasse diagrammer, sekvensdiagrammer o. lign. er lavet. Ulempen ved UML i forhold til resten af projektet, er at det er meget software centreret, og dermed mindre velegnet til beskrivelse af resten af projektet. 

4+1 modellen er en view model \footcite{viewmodel} designet til at beskrive software tunge systemer fra de forskellige stakeholder perspektiver. Modellen er brugt til at lave logic view, deployment view, implementation view og data view. Valget faldt på denne model, da systems kompleksitet i forhold til software blevet uoverskueligt lavet ved hjælp af UML diagrammer.

SysML står for System Modelling Language og er en udvidelse af UML, lavet med fokus på hardware og generelle system beskrivelser. SysML er brugt til at lave BDD’er, IBD’er Use Case diagrammer med mere, men ikke til softwaren. Dette skyldes at der i SysML er fjernet mange af de software specifikke ting som er i UML og at det derfor ikke er optimalt til beskrivelse af software.

Vores information om disse værktøjer kommer primært fra undervisningen i faget ISE\footcite{ise} på 2. semester, og fra Wikipedia. Den rent praktiske udførelse af diagrammerne er lavet i Microsoft Visio, da dette program er gratis tilgængeligt for studerende via Dreamspark.