% 3. Krav
\chapter{Krav}

Projektets aktører er brugeren, bevægelse i rummet og fysiske objekter under lampen. Brugeren er den primære aktør, og den der interagerer aktivt med produktet. Både bevægelse og fysiske objekter er sekundære aktører. Begge disse er sat som aktører da de aktiverer funktionalitet i produktet. Bevægelse registreres af en sensor, hvorefter lyset tændes. Fysiske objekter under lampen aktiverer, at lampen stopper nedadgående bevægelse, hvis lampen kommer for langt ned.
Vi har i projektet 5 Usecases (UC). De hedder følgende:

\begin{itemize}
    \item UC1: Kalibrér system.
    \item UC2: Tænd/Sluk lys.
    \item UC3: Registrér bevægelse.
    \item UC4: Juster lysets farve.
    \item UC5: indstil placering af lampen.
\end{itemize}

I UC1 kan brugeren, via systemets GUI, vælge at kalibrere systemet. Dette nulstiller lampens position i rummet, og gør efterfølgende placering af lampen mere præcis.

I UC2 kan brugeren, via systemets GUI, tænde og slukke for lampen. Dette gøres ved et simpelt tryk på en knap på touchskærmen.

I UC3 kan brugeren, via systemets GUI, aktivere funktionen ”registrér bevægelse”. Når denne funktion er aktiv, vil bevægelse i rummet resultere i at lyset tændes.

I UC4 kan brugeren, via systemets GUI, ændre på lysets farvetone. Dette gøres ved at indstille mængden af rødt, grønt og blåt lys lampen afgiver. Dette har også indvirkning på lysstyrken, da en kraftigere mængde af en eller flere farver vil give større lysstyrke.

I UC5 kan brugeren, via systemets GUI, vælge hvor i rummet lampen skal placeres. Dette sker ved at flytte på en slider for hhv. X, Y og Z aksen, og dermed indstille en position for lampen i alle bevægelses retningerne.

Alle aktører og Usecases inkl. tilhørende diagrammer, kan findes i dokumentationsrapporten\footcite{documentation} under punkt 2.2 Usecases.

Projektet er planlagt med en række ikke funktionelle krav inddelt under Usability, Reliability, Performance, Supportability og generelle krav. De ikke funktionelle krav definere under disse emner, systemets minimums levetid uden nedbrud og software’ens oppetid uden genstart. Derudover defineres betjeningsvenligheden, systemets responstid og almindelig vedligehold. De generelle krav fortæller hvad vægt tolerancen er for lampen, og hvad spændingstilslutning systemet kræver. Detaljerne kan findes i dokumentationsrapporten\footcite{documentation} under punkt 2.4 Ikke-funktionelle krav. 
