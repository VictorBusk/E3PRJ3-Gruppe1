% 7. Diskussion af resultater
\chapter{Diskussion af resultater}
Ud fra de gennemførte accepttests kan der med god samvittighed konstateres at prototypen på L.A.M.P. er en success.
Lampen kan bevæge sig fejlfrit i XY planen. I Z retningen er der tendens til at lampen ikke altid hejses fejlfrit, da der sidder en lille klump lim oppe under motoren som til tider kan løfte båndet op fra hjulet som trækker båndet rundt. Det kan i værste tilfælde få båndet til at glide af så lampen falder et par cm og snoren bliver viklet.

Hvad sensorerne angår, virker afstandssensoren og bevægelsessensoren i systemet. Afstandssensoren kan aktiveres og deaktiveres samt finjusteres via GUI'en. Bevægelsessensoren kan også aktiveres og deaktiveres via GUI'en, og er meget effektiv til at opfange bevægelse. Selv når man går rundt om prototypen kan bevægelsessensoren bemærke det.
Lyssensoren fungerer teknisk set i opsætningen. Dvs. den kan detektere et samlet lysniveau og systemet kan tolke dets målinger. Desværre ligger lysmålingerne der kommer fra diodernes lys, simpelthen så lave og ens, at der ikke kan tolkes noget ud fra om dioderne tilføjer noget markant til den samlede lysstyrke.

Lysdioderne virker helt som de skal; De kan tændes og slukkes, og deres farve kan ændres efter deres indstilling på touchskærmen. Selve lyset fra dioderne er dog ikke særlig stærkt i forhold til at de skulle fungere som lyskilder til et lokale.