% 4. Afgrænsning
\chapter{Afgrænsning}
\label{ch:afgraensning}

Når projektets use cases og funktionalitet er beskrevet, kan hardwareelementerne der skal implementeres fastlægges. Det er nødvendigt at have en fyldestgørende beskrivelse for de funktioner produktet skal kunne opfylde og udføre for at vælge de bedst mulige komponenter til at gennemføre dette. Grundet tidsrammen for projektet er det nødvendigt at begrænse sig for funktionalitet. Derfor vil det også være nødvendigt at fravælge nogle af de overvejede komponenter. Overvejelserne går på hvad der er vigtigst for produktets minimumsfunktionalitet.
Følgende produkter kunne implementeres i det endelige produkt:

\textbf{Kontakt (Tilvalgt)}
For at vognene i systemet ikke skulle køre ud over deres skinner er det nødvendigt at detektere skinnernes yderste position. Dette gøres ved en fysisk kontakt som slutter når vognen er i yderposition i enten X- eller Y-aksen. Derfor var en Kontakt nødvendig at have med. 

\textbf{Afstandssensor (Tilvalgt)}
For at vores lampe ville være i stand til at detektere sin bundposition var en form for sensor nødvendig at implementere her også. En afstandssensor ville være mere alsidig og give flere muligheder end en kontakt der blev monteret i lampeskærmen. Derfor blev afstandssensoren implementeret i produktet.

\textbf{Bevægelsessensor (Tilvalgt)}
I stedet for at tænde systemets lys med en kontakt, blev det valgt at en bevægelsessensor skulle benyttes. Blandt andet fordi endnu en kontakt ikke ville være en udfordring. Da der skulle trækkes flere løse ledninger til en kontakt og fordi udfordringen at arbejde med bevægelsessensoren var størst, blev denne sensor valgt.

\textbf{Lumensensor (Tilvalgt)}
For at give produktet funktionaliteten at automatisk kunne justere lysforholdene i rummet, måtte en lumensensor implementeres. Dette for blandt andet at kunne sætte et skema op for hvornår lyset skulle tændes eller ikke tændes, f.eks om morgenen i forhold til sommer-/vintertid. Desværre var produktets LED’er ikke kraftige nok til at give et på lumensensoren målbart arbejdsområde, så lumensensoren er ikke fuldt ud implementeret.

\textbf{Farvetemperatursensor (fravalgt)}
For at give produktet en ekstra feature der er en smule mere detaljeret end en almindelig lyssensor, var det muligt at implementere en farvetemperatursensor. At måle lysniveauer i rødt, grønt, og blåt lys fremfor blot at måle på hvidt lys ville være en forbedring af produktet da det nu også ville være et salgsargument for personer med visse øjensygdomme eller særlig følsomhed for bestemte lysspektre. Denne sensor blev fravalgt da den krævede et filter for kun at kunne måle det menneskelige synlige spektre. Uden dette filter målte den også infrarøde lysværdier. Grundet tidsrammen for projektet var der for meget arbejde i denne sensor, og den blev derfor fravalgt.  

\textbf{Trådløs styring (fravalgt)}
Produktet kunne udvides med trådløs betjeningspanel. Dette ville frigøre brugeren for at skulle tænke over hvor vedkommende skulle montere sit Devkit8000 i rummet. Da dette bliver betragtet som en feature der er unødvendig for produktets egentlige funktion, samt projektets tidsramme, blev denne funktionalitet fravalgt.  

\textbf{Gsm-modul (fravalgt)}
Lignende trådløs styring, kunne produktet styres trådløst via mobiltelefon. Dette ville frigøre nødvendigheden af et Devkit8000 og gøre brugeren i stand til at styre produktet uanset hvor han er, så længe det er inden for mobildækning. Dette blev også fravalgt grundet unødvendighed og tidsramme. 

