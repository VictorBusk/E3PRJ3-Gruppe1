% 0. Resumé
\chapter*{Resumé}

Denne rapport beskriver udarbejdelsen af Gruppe 1’s projekt i forbindelse med kurset PRJ3 som er semmester-projektet på 3. semester på ASE. Projektet har ifølge projektoplægget\footcite{prj3-oplaeg} skulle indeholde aktuatorer, brugergrænseflade, faglige elementer fra andre fag, indlejret Linux- og en PSoC platform.

Den valgte projektformulering er at udvikle et automatisk lampe-hejsesystem ’L.A.M.P.’ der skal give mulighed for at finjustere en belysning. Dette hejse-system skal placeres under loftet i et lokale og skal styres fra en brugergrænseflade i form af en Devkit8000 touchskærm. Lampen skal kunne bevæge sig i 3 dimensioner: en X-akse og Y-akse (på langs og tværs af lokalet), samt en Z-akse (i højden). Lampen kan altså kunne placeres i en vilkårlig position i et hvert lokale. 3 typer sensorer skal derudover kunne detektere hhv. lysniveau i rummet, afstand fra lampe ned til nærmeste forhindring og om der er bevægelse i lokalet. 

Projektet er udført efter ASE-modellen\footcite{udviklingVejledning} og en prototype er blevet designet, konstrueret og implementeret. Projektets underemner, dvs. Projektformulering, Kravspecifikation, Design og Implementering er udarbejdet iterativt.

En række accepttest af funktionaliteten for L.A.M.P. er lavet\footcite{documentation} og gennemført til at dokumentere at hele systemet er operativt og virker efter hensigten.
