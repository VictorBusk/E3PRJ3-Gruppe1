% 0. Resumé / Abstract
\chapter*{Resumé / Abstract}

\section{Resumé}

Denne rapport beskriver udarbejdelsen af Gruppe 1’s projekt i forbindelse med kurset PRJ3 som er semmester-projektet på 3. semester på ASE. Projektet skulle indeholde aktuatorer, brugergrænseflade, faglige elementer fra andre fag, indlejret Linux- og en PSoC platform.

Den valgte projektformulering er at udvikle et automatisk lampe-hejsesystem ’L.A.M.P.’ der skal give mulighed for at finjustere en belysning. Dette hejse-system skal placeres under loftet i et lokale og skal styres fra en brugergrænseflade i form af en Devkit8000 touchkærm. Lampen skal kunne bevæge sig i 3 dimensioner: en X-akse og Y-akse (på langs og tværs af lokalet), samt en Z-akse (i højden). Lampen kan altså kunne placeres i en vilkårlig position i et hvert lokale. 3 typer sensorer skal derudover kunne detektere hhv. lysniveau i rummet, afstand fra lampe ned til nærmeste forhindring og om der er bevægelse i lokalet. 

Projektet er udført efter ASE-modellen og en prototype er blevet designet, konstrueret og implementeret. Projektets underemner, dvs. Projektformulering, Kravspecifikation, Design og Implementering er udarbejdet iterativt.

En række accepttest af funktionaliteten for L.A.M.P. er lavet og gennemført til at dokumentere at hele systemet er operativt og virker efter hensigten.

\section{Abstract}

This report describes the compiled project of Group 1 regarding the course PRJ3 which is the semester-project of the 3rd semester at ASE. The project was required to contain operators, a user interface, topics from other courses, embedded Linux and a PSoC platform.

The chosen project formulation is to develop an automatic lamp-crane-system ‘L.A.M.P.’ that will enable the lamps position to be easily customized. The crane-structure will be placed under the ceiling in a room and will be controlled from an accessible user interface by means of a Devkit8000 touchscreen. The lamp will be able to be moved in 3 dimensions: An X- and Y-axis (along and across the room) and a Z-axis (at height). The lamp is able to be positioned freely in a limited 3-dimensional space. 3 different types of sensors will also respectively detect the brightness in the room, the distance from the lamp down to nearest obstacle and whether there is any movement in the room or not.

The Project is completed trough the ASE-model and a prototype has been designed, constructed and implemented. The subtopics of the project, that is, Project formulation, Requirements-specifications, design and implementation are created iteratively.

A series of Acceptance-tests of the functionality for L.A.M.P is conducted and completed in order to ensure the total system is operational and works as intended.
