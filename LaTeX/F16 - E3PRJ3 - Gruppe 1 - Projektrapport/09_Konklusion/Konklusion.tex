% 9. Konklusion
\chapter{Konklusion}

Formålet med opgaven var at udvikle et system der involverede, sensorer/aktuatorer, en brugergrænseflade, en indlejret Linux platform, en PSoC platform og skulle indeholde faglige elementer fra semestrets fag. Alle disse kriterier opfyldes i dette projekt. En indlejret Linux platform blev brugt som brugergrænseflade ved at agere som systemets controller styret fra et grafisk brugerinterface. PSoC platformer er blevet brugt både som kommunikationscentral (beskedcentral) til distribuering af data, til stepper motorstyring og behandling af sensorinput.


De fastlagte krav der var essentielle for acceptabel produktfunktionalitet virkede og blev alle godkendt ved test. Lampens primære formål var at være en modificerbar lysgiver. Disse modificerbare features bestod af henholdsvis muligheden for at indstille lysets farve, samt placeringen af lampen. Disse funktioner fungerede og kunne indstilles direkte fra GUI’en. 
Ligeledes var nogle sensorer en del af lampekonstruktionen, og har til formål at give en generel god brugeroplevelse. Afstandssensoren blev brugt til at undgå kollision med diverse objekter ved vertikal bevægelse, og til at tilpasse kalibreringen. Afstandssensoren var fuldt funktionel, præcis og dens indstillinger kunne ændres live fra GUI’en og var derfor tilfredsstillende. 
Bevægelsessensorens, såvel som lumensensorens indstillinger, kunne ligeledes ændres live. Dette blev gennem tests bekræftet at være funktionelt. Stepmotorerne og kodningen af dem forløb ligeledes ufejlbarlig ved test, da de altid bevægede sig korrekt i forhold til de modtagede input. Det samlede system som helhed fungerede og kommunikationen internt virkede.


Problemerne med at lumensensoren ikke gav systemet særligt brugbare data, resulterede i at på trods af at systemet kunne læse værdier fra lysinputtet, lå kvaliteten af dette input på så lavt et niveau, at det ikke ville give mening at benytte i resten af systemet. Tilpasning af lysstyrke i forhold til de omkringværende lysforhold blev af denne årsag ikke en realitet. Dette problem blev opdaget så sent i processen, at der var enighed om at se bort fra dette element af produktet og sætte fokus på resten af produktets færdiggørelse.
En \textit{could have} blev dog tilføjet til GUI’en: Muligheden for at brugeren kunne gemme farveindstillinger såvel som placering af lampen blev realiseret.
