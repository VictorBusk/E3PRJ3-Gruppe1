% 9. Konklusion
\chapter{Konklusion}

Formålet med opgaven var at udvikle en system der involverede, sensorer/aktuatorer, en brugergrænseflade, en indlejret Linux platform, en PSoC platform og skulle indeholde faglige elementer fra semestrets fag. Alle disse kriterier opfyldes i dette projekt. En indlejret Linux platform blev brugt som brugergrænseflade ved at agere som systemets controller styret fra et grafisk brugerinterface. PSoC platformer er blevet brugt både som kommunikationscentral (postkontor) til distribuering af data og til stepmotorstyring samt behandling sensorinput.


De fastlagte hårde krav der var fastsat som essentielle for acceptabel produktfunktionalitet virkede og blev alle godkendt ved test. Lampens primære formål var at være en modificerbar lysgiver. Disse modificerbare features bestod af henholdsvis muligheden for at indstille lysets farve og placeringen af lampen. Disse funktionen fungerede fejlfrit og kunne indstilles direkte fra GUI’en. Ligeledes var nogle sensorer ligeledes en del af lampekonstruktionen som har til formål at give en generel god brugeroplevelse. Afstandssensoren der blev brugt til at undgå kollision med diverse objekter der kunne stå som forhindring, og blev ligeledes benyttet til at tilpasse kalibreringen. Afstandssensoren var fuldt funktionel, præcis og dens indstillinger kunne ændres live fra GUI’en og var derfor tilfredsstillende. Bevægelsessensorens, såvel som lumensensorens indstillinger kunne ligeledes ændres live. Hvilket ved test blev bekræftet at være funktionelt. Stepmotorerne og kodningen af dem forløb ligeledes ufejlbarlig ved test da de altid bevægede sig korrekt i forhold til de modtaget inputs. Systemet som helhed fungerede som samlet kredsløb og kommunikationen internt virkede.


Problemerne med at lumensensorens data ikke gav systemet særligt brugbare data resulterede i at systemet kunne læse værdier fra lysinputtet, men kvaliteten af den modtaget data lå ikke på et højt nok niveau til at det ville give mening at benytte i resten af systemet. Tilpasning af lysstyrke i forhold til de omkringværende lysforhold blev af denne årsag ikke en realitet. Dette problem blev opdaget så sent i processen at der var enighed om at skrotte dette element af produktet.
En could have, blev dog tilføjet til GUI’en. Muligheden for at brugeren kunne gemme farveindstillinger såvel som placering af lampen blev realiseret. 