\section{ATUC1: Kalibrér system}

\begin{center} \centering
    \begin{longtable}{|p{0,5cm}|p{3,7cm}|p{3,7cm}|p{2,3cm}|p{2,3cm}|}
    \hline
        \multicolumn{5}{|l|}{\textbf{ATUC1: Kalibrer system}} \\ \hline
        \multicolumn{1}{|c|}{} &
        \textbf{Test} &
        \textbf{Forventet \newline Resultat} &
        \textbf{Resultat} &
        \textbf{Godkendt\slash \newline Kommentar} \\ \hline 
        \endfirsthead

        \multicolumn{5}{l}{...fortsat fra forrige side} \\ \hline 
        \multicolumn{5}{|l|}{\textbf{ATUC1: Kalibrer system}} \\ \hline
        \multicolumn{1}{|c|}{} &
        \textbf{Test} &
        \textbf{Forventet \newline Resultat} &
        \textbf{Resultat} &
        \textbf{Godkendt\slash \newline Kommentar} \\ \hline 
        \endhead
        
        \multicolumn{5}{r}{fortsættes på næste side...} \\
        \endfoot
        \endlastfoot

        \textbf{1} 
            & Brugeren trykker ”Calibrate”-knap på touchskærmen under Settings-fanebladet.
            & Lampen trækkes helt op og bliver så kørt helt frem og tilbage i først X-dimensionen og dernæst Y-dimensionen. Disse kalibreringsværdier forventes returneret og gemt til de pågældende PSoC'er.
            & Kalibrerings-
            rutinen blev tilendebragt i alle tre akser fejlfrit og kalibreringsdata blev gemt.	
            & Godkendt. 
        \\ \hline
	\end{longtable}
	\label{ATUC1} 
\end{center}