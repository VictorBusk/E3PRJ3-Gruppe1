\subsection{Signalbeskrivelse}

For at fuldende beskrivelsen af grænsefladen er der lavet en signaltabel \ref{tab:signaltabel}. Hvert signal er beskrevet, og området et signal er defineret under, er også beskrevet. Blok og terminal indgår også. 
\begin{center} \centering
    \begin{longtable}{|p{2,6cm}|p{2,6cm}|p{2,6cm}|p{2,6cm}|p{2,6cm}|}\hline
	\textbf{Signalnavn} & \textbf{Funktion} & \textbf{Område} & \textbf{Port 1} & \textbf{Port 2} \\ \hline
	\endfirsthead
		
	\multicolumn{5}{l}{...fortsat fra forrige side} \\ \hline 
	\textbf{Signalnavn} & \textbf{Funktion} & \textbf{Område} & \textbf{Port 1} & \textbf{Port 2} \\ \hline
	\endhead
	
	\multicolumn{5}{r}{fortsættes på næste side...} \\
    \endfoot
    \endlastfoot

        I2C
        & Seriel 2-wire kommunikation
        & 
        & PSoC-Master, PSoC-Z
        & PSoC-XY, PSoC-Z, PSoC-Sensor, Lumen sensor
        
        \\ \hline 
        
        TTL
        & Højt/Lavt signal
        & 5V
        & PSoC-Sensor
        & PIR og Afstandssensor
        
        \\ \hline 
        
        SPI
        & Seriel 4-wire kommunikation
        &
        & Devkit8000
        & PSoC-Master 
        \\ \hline 
        
        5V
        & Forsyning til motor
        & 0V til 5V
        & PSoC-XY, PSoC-Z
        & Motor
        \\ \hline
        
        Max 3V
        & Forsyning til LED
        & 0V til 3V
        & PSoC-Sensor
        & LED
        \\ \hline
        
        5V
        & Forsyning til PSoC
        & 5V
        & Strømforsyning
        & PSoC-XY 
            \newline PSoC-Z
            \newline PSoC-Sensor
        \\ \hline
        
        On/Off
        & Bryde eller afbryde et signal
        & On/Off
        & PSoC-XY
            \newline PSoC-Z
        & PSoC-XY 
            \newline PSoC-Z
        \\ \hline
	\caption{Signaltabel}
	\label{tab:signaltabel} 
    \end{longtable}
\end{center}