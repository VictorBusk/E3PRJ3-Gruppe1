\section{Protokol}

En del data skal flyttes mellem DevKit, PSoC-Master med SPI og mellem PSoC-Master og PSoC-XY, PSoC-Z og PSoC-Sensor med I2C. Her følger beskrivelsen af hvordan kommunikationen over hhv. SPI og I2C forgår.

\subsection{SPI}

Opsætning er som følger:
\begin{itemize}
    \item Hastighed: 1 MHz
    \item SPI mode: 3 (CPOL 1 - CPHA 1)
    \item Antal bits: 16
\end{itemize}

Hastigheden er valgt på baggrund af Hal Exercise7\footnote{Hardware abstraktioner. Exercise 7: LDD with SPI. Øvelse med SPI Kommunikation}, PSoC kan dog køre med op til 8 MHz, men der er valgt en lavere hastighed da det vil give en større stabilitet.

\subsubsection{Kommandoer til SPI}

\subsection{I2C}

\begin{table}[H]
\caption{Master sender en commando og tilhørende værdi}
\centering
\begin{tabular}{|c|c|c|c|c|c|c|c|c|c|c|c|c|c|c|c|c|c|}
\hline 
\multicolumn{8}{|c|}{1. Byte} & ACK & \multicolumn{8}{|c|}{2. Byte} & ACK \\
\hline 
\textbf{7} & \textbf{6} & \textbf{5} & \textbf{4} & \textbf{3} & \textbf{2} & \textbf{1} & \textbf{0} & \textbf{} & \textbf{7} & \textbf{6} & \textbf{5} & \textbf{4} & \textbf{3} & \textbf{2} & \textbf{1} & \textbf{0} & \textbf{} \\
\hline 
MSB & & & & & & & LSB & & \multicolumn{8}{|c|}{Most signifiant  byte} & \\ 
\hline
\multicolumn{8}{|c|}{1. Byte send to slave} & by & \multicolumn{8}{|c|}{2. Byte send to slave} & by \\
\hline
\multicolumn{7}{|c|}{Adress slave} & R/W & slave & \multicolumn{8}{|c|}{Command send to slave} & slave\\
\hline \hline
\multicolumn{8}{|c|}{3. Byte} & ACK & \multicolumn{8}{|c|}{4. Byte} & ACK \\
\hline
\textbf{7} & \textbf{6} & \textbf{5} & \textbf{4} & \textbf{3} & \textbf{2} & \textbf{1} & \textbf{0} & \textbf{} & \textbf{7} & \textbf{6} & \textbf{5} & \textbf{4} & \textbf{3} & \textbf{2} & \textbf{1} & \textbf{0} & \textbf{} \\
\hline
\multicolumn{8}{|c|}{} & & \multicolumn{8}{|c|}{} & \\
\hline
\multicolumn{8}{|c|}{Data send to slave} & by & \multicolumn{8}{|c|}{Data send to slave} & by \\
\hline
\multicolumn{8}{|c|}{3. Byte send to slave} & slave & \multicolumn{8}{|c|}{4. Byte send to slave} & slave \\
\hline \hline
\multicolumn{8}{|c|}{5. Byte} & ACK & \multicolumn{8}{|c|}{} & \\
\hline
\textbf{7} & \textbf{6} & \textbf{5} & \textbf{4} & \textbf{3} & \textbf{2} & \textbf{1} & \textbf{0} & \textbf{} & \multicolumn{8}{|c|}{} & \textbf{} \\
\hline
\multicolumn{8}{|c|}{Least signifiant  byte} & & \multicolumn{8}{|c|}{} & \\
\hline
\multicolumn{8}{|c|}{Data send to slave} & by & \multicolumn{8}{|c|}{} & \\
\hline
\multicolumn{8}{|c|}{5. Byte send to slave} & slave & \multicolumn{8}{|c|}{} & \\
\hline
\end{tabular}
\label{tabel:I2CMasterData}
\end{table} 

\begin{table}[H]
\caption{2. Master modtager data}
\centering
\begin{tabular}{|c|c|c|c|c|c|c|c|}
\hline 
\textbf{7} & \textbf{6} & \textbf{5} & \textbf{4} & \textbf{3} & \textbf{2} & \textbf{1} & \textbf{0}\\ 
\hline 
MSB & & & & & & & LSB \\ 
\hline
\multicolumn{8}{|c|}{Byte read from slave} \\
\hline
\end{tabular}
\label{tabel:I2CMasterCommando}
\end{table} 

Efter hver modtaget byte sender masteren en ACK til slaven ved at sætte SDA lav og generere en clock-puls. Herefter overtager slaven igen SDA.
Efter sidste modtaget byte, sender masteren i stedet STOP ved at sætte SCL høj og derefter SDA høj og bussen er derefter frigivet igen.