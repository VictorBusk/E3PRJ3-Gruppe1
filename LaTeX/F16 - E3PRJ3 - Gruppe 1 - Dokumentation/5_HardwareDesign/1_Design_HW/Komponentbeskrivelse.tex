\subsection{Komponentbeskrivelser}

\subsubsection{Lumen sensor}

Lumen sensoren der er valgt til projektet kommunikerer over I2C, og har dermed de fire standard ben: VDD og GND til forsyning samt SDA og SCL til I2C kommunikationen. Derudover har den to ekstra ben: ADDR-SEL som bruges til at vælge mellem tre forskellige addresser, så konflikter kan undgås, og INT som sensoren kan bruge til at sende en interrupt, hvis den funktionalitet bliver aktiveret.

Forsyningsspændingen skal ligge mellem 2.7 og 3.6 V, og sensoren bruger max 0.6 mA.

I kommunikationen med sensoren er der brug for to generelle funktioner:

\begin{enumerate}
    \item Skrive noget kontroldata til sensoren for at justerer dens opførsel
    \item Læse data fra sensoren for enten at verificerer at kontroldata er korrekt modtaget, eller at aflæse målinger
\end{enumerate}

Formatet for funktionerne er (grå felter bliver sendt af sensoren):

\begin{figure}[H] \centering
    \includegraphics{0_Filer/Figuer/5_HW_Design/Lumen_skriv.png}
    \caption{Skrive kontroldata til sensor}
    \label{fig:HWD_Lumen_skriv}
\end{figure}

\begin{figure}[H] \centering
    \includegraphics[width=\textwidth]{0_Filer/Figuer/5_HW_Design/Lumen_laes.png}
    \caption{Læse data fra sensor}
    \label{fig:HWD_Lumen_laes}
\end{figure}

Når sensoren først tændes starter den i power-down mode, derefter kan sensor funktionaliteten startes ved at sende en I2C kommando til den (kommando: 0x80 og data: 0x03). Efter sensoren er blevet sat til at køre kan dens to kanaler aflæses med I2C (kommando: 0xAC eller 0xAE).

De to output kanaler giver lysmålinger i det synlige + infrarøde område (kanal 0), og det infrarøde område (kanal 1). Så de to aflæsninger skal kombineres  for at få et tal kun for mængden af det synlige lys.

\begin{figure}[H] \centering
    \includegraphics{0_Filer/Figuer/5_HW_Design/Kombinerings_formler.png}
    \caption{Kombinerings formler}
    \label{fig:HWD_Lumen_formler}
\end{figure}

\subsubsection{Afstandssensor}

Den valgte afstandssensor til projektet er af typen HC-SR04. Sensoren er en ultralyds sensor, som bruger sonar til kontaktfri afstandsmåling. Sensorens arbejdsfrekvens ligger på 40 kHz og output er et TTL signal.

Sensoren har 4 ben, VCC, GND, Trig og Echo som vist på billedet.

\begin{figure}[H] \centering
    \includegraphics{0_Filer/Figuer/5_HW_Design/HC-SR04.jpg}
    \caption{HC-SR04 afstandssensor}
    \label{fig:HWD_SR04}
\end{figure}

Sensoren har en opløsning på 0,3 cm og er stabil og præcis i intervallet 2 cm til 400 cm, og måler i en vinkel på 15$^{\circ}$. Sensoren skal aktiveres med en TTL puls på mindst 10 mikrosekunder ind på Trig benet. Output fra sensoren kommer som et TTL signal på Echo benet, der har samme længde som tiden fra udsendt sonar burst til ekkosignal modtages af sensoren.

HC-SR04 skal forsynes med 5 VDC, har en idle current på under 2 mA og en working current på 15 mA.

Omregning af output fra sensoren til afstand i cm gøres på følgende måde:

$$Dist(cm) = (\frac{HighLevelTime(s)*340(\frac{m}{s})}{2})*100$$

Forhold mellem trigger TTL puls, sonar burst og TTL ekko kan ses på Timing diagrammet herunder.

\begin{figure}[H] \centering
    \includegraphics[width=\linewidth]{0_Filer/Figuer/5_HW_Design/HC-SR04_Ultra_Sonic_Timing_Diagram.PNG}
    \caption{HC-SR04 timing diagram}
    \label{fig:HWD_SR04_timing}
\end{figure}

\subsubsection{Bevægelses sensor (PIR)}

Den valgte PIR sensor til projektet er af typen HC-SR501. Sensoren har 3 ben -  VCC, GND og et digitalt output ben. HC-SR501 skal have en forsyningsspænding på mellem 5 og 20 VDC og bruger 65mA. Det digitale output er med TTL standard, hvor High = 3,3V og Low = 0V. HC-SR501 har en justerbar forsinkelse på mellem 0,3 og 5 min. Samt mulighed for at justere max-sensor range fra 3 til 7 m. Både delay og range stilles ved hjælp af hver deres potentiometer.

Ved opstart af modulet vil modulet udsende TTL high 0-3 gange i løbet af det første minut, og derefter gå i standby. Herefter er modulet klar.

Der er mulighed for at sætte sensoren op i to modes ved hjælp af en jumper. Non-repeatable trigger og Repeatable trigger.

Non-repeatable trigger: Når sensoren aktiveres går output benet høj i det interval, der er indstillet med delayet, derefter går benet lavt igen. Benet bliver ikke holdt højt ved gentagen bevægelse, men kan først trigges igen, når det er gået lavt.

Repeatable trigger: Når sensoren aktiveres går output benet højt, og så længe der fortsat er bevægelse vil benet forblive højt. Når bevægelse ophører, vil benet forblive højt til delayet er slut og derefter gå lavt.

\begin{figure}[H]
\centering
\begin{minipage}{.6\textwidth}
  \centering
    \includegraphics[width=\linewidth]{0_Filer/Figuer/5_HW_Design/HC-SR501_PIR_Motion_Detector_image.png}
    \caption{HC-SR501 PIR sensor}
    \label{fig:HWD_SR501}
\end{minipage}%
\begin{minipage}{.4\textwidth}
  \centering
    \begin{tabular}{ | l | l | }
        \hline
        \textbf{PSoC} & \textbf{HC-SR501} \\ \hline
        GND & GND \\ \hline
        5 V & Vcc \\ \hline
        P2\textunderscore{}0 & TTL \\
        \hline
    \end{tabular}
    \caption{Portforbindelser}
    \label{fig:HWD_PIR_ports}
\end{minipage}
\end{figure}

\subsubsection{5V Stepper motor}

\begin{figure}[H]
\centering
\begin{minipage}{.5\textwidth}
  \centering
  \includegraphics[width=\linewidth]{0_Filer/Figuer/5_HW_Design/Stepper_motor.png}
  \caption{5V stepper motor}
  \label{fig:HWD_Stepper_Motor}
\end{minipage}%
\begin{minipage}{.5\textwidth}
  \centering
  \includegraphics[width=\linewidth]{0_Filer/Figuer/5_HW_Design/Motor_diagram.png}
  \caption{Intern forbindelse i motoren}
  \label{fig:HWD_Stepper_Diagram}
\end{minipage}
\end{figure}

Til styring af X,Y og Z retning benyttes en 5V stepper motor af typen 28BYJ-48. Motoren er et oplagt valg, siden den er let tilgængelig på embedded stock, og da den lever op til de krav, som er til systemet. Yderligere er 28BYJ-48-motoren mere kompatibel med det størrelsesforhold prototypen bygges i. Der udover er motorens præcision fordelagtig for projektet, da den har en gearing på 64x48. Dvs. at der er 3072 steps pr. omgang, som gør at lampen kan positioneres meget præcist.

\begin{figure}[H] \centering
    \fbox{\includegraphics[width=.6\textwidth]{0_Filer/Figuer/5_HW_Design/Bipolar_motor_diagram.png}}
    \caption{Stepper motor}
    \label{fig:HWD_Bipolar_Stepper_Motor}
\end{figure}

Forklaringen herunder tager udgangspunkt i Stator A.
\newline Motoren består af 2 statorer med en nord- og en sydpol. Hver stator har viklinger med to ende tilslutninger, phase A og phase A’, mellem de to ende tilslutninger tilsluttes en 5VDC spænding(Vm). Se figur \ref{fig:HWD_Bipolar_Stepper_Motor}.
Ved at tilslutte phase A til stel, vil strømmen bevæge sig fra Vm til phase A gennem viklingerne, denne bevægelse vil danne et magnetfelt, hvor nord og sydpolen er som på figur \ref{fig:HWD_Bipolar_Stepper_Motor}. Ved tilslutning af phase A’ vil strømmen bevæge sig i modsat retning og de to poler vil bytte plads. Dette gælder også for stator B. Når polerne skifter vil rotorens poler i motoren blive henholdsvis tiltrukket og afvist af statorens poler.

\subsubsection{Micro switch}

\begin{figure}[H]
\centering
\begin{minipage}{.5\textwidth}
  \centering
  \includegraphics[width=\linewidth]{0_Filer/Figuer/5_HW_Design/Micro_switch.png}
  \caption{Micro Switch (DM1-01P-30-3)}
  \label{fig:HWD_Micro_switch}
\end{minipage}%
\begin{minipage}{.5\textwidth}
  \centering
  \includegraphics[width=\linewidth]{0_Filer/Figuer/5_HW_Design/Micro_switch_diagram.png}
  \caption{Forbindelses diagram af micro switch}
  \label{fig:HWD_Micro_Diagram}
\end{minipage}
\end{figure}

Til håndtering af ende stop og kalibrering af X-,Y- og Z-retninger, installeres 5 micro switches(DM1-01P-30-3). To switches placeres i enderne på X-skinnen, to på Y-skinnens ender og en ved lampens top-position. Ved aktivering af en af micro switches sættes en port høj på den tilhørende PSoC4, som vil stoppe den bevægelse der nu er nået enden.
Micro switchen forbindes som en NO(normaly open) switch med 5V på COM benet og NO benet forbundet til PSoC4. Se \ref{fig:HWD_Micro_Diagram}.
Iflg. Databladet er micro switchen rated 5V DC og 30mA.

\subsubsection{Strømforbrug}

En enkelt motor bruger 150 mA, når den ikke drejer. Strømforbruget falder når motoren kører. PSoC-XY kommer derfor til at levere mest strøm, da den forsyner 4 motorer, altså 4x 150 mA, og dermed 600 mA hvilket ikke er noget problem for en PSoC. PSoC-Sensor vil levere næstmest strøm i størrelsesordenen 260 mA, beregnet samlet ud fra de forskellige sensorer og LED’er som skal drives af PSoC’en. PSoC-Z levere kun strøm til én motor, hvilket bliver 150 mA.

\subsubsection{Motorstyring}

Et styringsprint er lavet til positionering af vores stepper-motorer. Dette er valgt for at lave en samlet printboks bestående af prints og PSoC’s i vores system. På denne måde kan diverse ledninger, der ellers ville ligge rodet oven i hinanden, blive pakket kompakt i en mere æstetisk behagelig kasse.

\begin{figure}[H] \centering
    \fbox{\includegraphics[width=\textwidth]{0_Filer/Figuer/5_HW_Design/X-Diagrammet.PNG}}
    \caption{Styringsprint til X-akse og 34-leder}
    \label{fig:HWD_Styringsprint_til_x_print}
\end{figure}

Kommunikationen går fra de styrende prints gennem et 34-lederkabel og ud til aktuatorerne. På Figur \ref{fig:HWD_Styringsprint_til_x_print} ses et diagram for udlægget af 34-lederkablet. I forsyningskredsløbet sendes der et input ind fra den styrende PSoC, som da bliver forstærket igennem en MosFet. Herefter bliver det sendt ud på 34-lederkablet og går op i et splitterprint i vognene på X- og Y-aksen. Fra disse splitterprints, går signalet direkte ud til steppermotorerne. 


\textbf{Styringskode:}\newline 
Styringen af systemets motorer foregår i tildelte PSoCs. Hver bevægelsesretning har en PSoC tildelt som styre retning og antal trin motoren skal køre. Aktivering af motorerne kan initieres i PSoC'en på to måder. Manual-mode og Automatic-mode. Der skiftes mellem disse to modes ved tryk på switch på PSoC. Nuværende mode indikeres gennem farven på LED.

\textbf{Manual-Mode:} Indikeres med farven rød på LED på PSoC.\newline 
Når LED’en på PSoC’en lyser rød, er I2C kommunikation med Master-PSoC slået fra, og Capsense slideren på PSoC’en er slået til. 
Slideren er delt op i fire felter, fra 0 til 128, til manuel bevægelse af motorerne.  Felt 1 er alle værdier under 32. felt 2 er værdier mellem 32 og 64. felt 3 er værdier mellem 64 og 96. felt 4 er værdier over 96.
Disse fire felter styrer hver deres bevægelse af motorerne. Felt 1 er Y-motorerne i den ene retning, felt 2 er Y-motorerne i modsat retning af Felt 1. Felt 3 er X-motorerne tilsvarende i den ene retning, og Felt 4 er modsat retning af Felt 3. 
For at aktivere disse retningskørsler trykkes der på et givent felt på Capsense-slideren, og fingeren holdes i kontakt med dette felt, så længe kørsel med motorerne ønskes. Motorerne stoppes når kontakt med felt afbrydes.\newline 

Oversigt over felterne:\newline
\verb+Kør_X_frem();+       mellem 64 og 96 på touch-sense\newline
\verb+Kør_X_tilbage();+      over 96 på touch-sense\newline
\verb+Kør_Y_frem();+	   mellem 64 og 32 på touch-sense\newline
\verb+Kør_Y_tilbage();+	     under 32 på touch-sense\newline

\textbf{Automatic-Mode:} Indikeres med farven grøn på LED på PSoC.\newline 
Når LED’en på PSoC’en lyser grøn, er Capsense slideren på PSoC’en slået fra, og I2C kommunikation med Master-PSoC er slået til. 
Ved Automatic-mode modtager PSoC’en til styring af XY-retningen en kommandokode fra PSoCMaster. Denne kommandokode kommer via I2C gennem  P4[0] til SCL og P4[1] til SDA på PSoC-XY.
Kommandokoden identificeres til et funktionskald og en parameter. For at køre med motorerne, kræver det at det første funktionskald, der bliver givet er kalibrer\textunderscore{}system. 

\textbf{Styringskode opbygning:}\newline
For overskuelighed er koden delt op i nogle header- og source filer se \ref{fig:XYhs} 

% XY header/source filer
\begin{figure}[H] \centering
    \includegraphics[width=0.3\linewidth]{0_Filer/Figuer/5_HW_Design/HeaderSourceOversigt.PNG}
    \caption{XY header/source filer}
    \label{fig:XYhs}
\end{figure}

\verb+cyapicallbacks:+\newline
Denne del er koden sørger for at når der kommer en kommando på I2C, kalder den et interupt, der stopper programmet og ser på hvilken kommando der er kommet. Efter følgene hopper den tilbage hvor den kom til i programmet og fortsætter med de eventuelt nye parametre der er kommet. \newline
\verb+data:+\newline

\verb+handler:+\newline
\verb+i2c:+\newline
\verb+led:+\newline
\verb+queue:+\newline
\verb+xy:+\newline
\verb+main:+\newline

%Venter med resten for at se hvad jeppe skriver

\textbf{Kalibrer\textunderscore{}system()}\newline 
Dette funktionskald aktiverer X-motorerne i den ene retning. Når retningens grænse er nået, vil en fysisk afbryder blive aktiveret og herved leverer besked til PSoC-XY om at den nedre grænse er nået. Herefter vil X-motorerne køre i den anden retning til de rammer den øvre grænse hvor endnu en afbryder bliver aktiveret. \newline 
De steps motoren tager imellem den nedre og øvre grænse bliver gemt i en integer variabel kaldet X-max. Motorens nuværende position ved den øvre grænse gemmes i en integer variabel kaldet X-position som bliver sat til 0. \newline 
Når X-max og X-position er sat, aktiveres Y-motorerne i den ene retning. Når retningens grænse er nået, vil en fysisk afbryder blive aktiveret og herved leverer besked til PSoC-XY om at den nedre grænse er nået. Herefter vil Y-motorerne køre i den anden retning til de rammer den øvre grænse hvor endnu en afbryder bliver aktiveret. \newline 
Her bliver antal steps Y-motorerne har kørt gemt i en integer variabel kaldet Y-max, og motorens nuværende position bliver gemt i en integer variabel kaldet Y-position som sættes til 0. 
Hermed er skinnernes længde målt ud, lampens position er sat og systemet er nu kalibreret.   \newline  
\textbf{Get\textunderscore{}X()}\newline 
Denne funktion returnerer motorens position fra 0 til 255 fordelt ud på X-max værdien.  \newline 
\textbf{Set\textunderscore{}X()}\newline 
Denne funktion aktivere X-motorerne. Funktionen giver en parameter fra 0 til 255. Denne parameter bruges til at sætte den ønskede position som motorerne skal køre til. Hvis ønskede position er det samme som nuværende position, køres der ikke med motorerne. Alt efter hvilken position der bliverønsket fra PSoC-Master, regner PSoC-XY selv ud hvilken retning motorerne skal køre for at opnå den ønskede position. \newline  
\textbf{Get\textunderscore{}Y()}\newline 
Denne funktion returnerer motorens position fra 0 til 255 fordelt ud på Y-max værdien.  \newline 
\textbf{Set\textunderscore{}Y()}\newline 
Denne funktion aktivere Y-motorerne. Funktionen giver en parameter fra 0 til 255. Denne parameter bruges til at sætte den ønskede position som motorerne skal køre til. Hvis ønskede position er det samme som nuværende position, køres der ikke med motorerne. Alt efter hvilken position der bliver ønsket fra PSoC-Master, regner PSoC-XY selv ud hvilken retning motorerne skal køre for at opnå den ønskede position.  \newline 
