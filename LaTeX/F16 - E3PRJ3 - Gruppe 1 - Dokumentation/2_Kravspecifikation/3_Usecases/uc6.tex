\subsection{UC6: Indstil placering i højden}

\begin{center} \centering
	\begin{longtable}{|p{6cm}|p{8cm}|}
	\hline
		\multicolumn{2}{|l|}{\textbf{UC6: Indstil placering i højden}} \\\hline
		\endfirsthead
		
		\multicolumn{2}{l}{...fortsat fra forrige side} \\ \hline 
		\multicolumn{2}{|l|}{\textbf{UC6: Indstil placering i højden}} \\\hline
		\endhead		

        \multicolumn{2}{r}{fortsættes på næste side...} \\
        \endfoot
        \endlastfoot
        
        \textbf{Mål}								
            & Brugeren får hævet eller sænket lampen til den valgte højde.
        \\ \hline
        \textbf{Initialisering}					
            & Bruger trykker på touchskærmen.
        \\ \hline
        \textbf{Aktører og Stakeholders}			
            & Bruger.
        \\ \hline
        \textbf{Referencer}						
            & Ingen.
        \\ \hline
        \textbf{Antal af samtidige hændelser}	
            & Ingen.
        \\ \hline
        \textbf{Forudsætning}					
            & At systemet er tændt, kalibreret og funktionsdygtigt.
        \\ \hline		
        \textbf{Efterfølgende tilstand}	
            & Lampens højde er indstillet til det valgte niveau.
        \\ \hline
        \textbf{Hovedforløb}						
            &
            \begin{enumerate}
                \item Bruger ændre Z-slider positionen, i position tappen.
                \item Brugeren trykker på GO-kanppen, på position tappen. 
                    \newline [Undtagelse 1: Sensor registrerer forhindring]
                \item Lampe er nu indstillet til den valgte højde.
            \end{enumerate}
        \\ \hline
        \textbf{Undtagelser}						
            & [Undtagelse 1: Sensor registrerer forhindring]
            \begin{enumerate}
                \item Systemet stopper nedsænkningen af lampen og kan kun køre opad igen.
            \end{enumerate}
        \\ \hline
	\end{longtable}
	\label{UC7} 
\end{center}