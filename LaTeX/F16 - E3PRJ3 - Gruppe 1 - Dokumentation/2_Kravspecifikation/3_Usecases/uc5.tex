\subsection{UC5: Indstil placering i af lampen}

\begin{center} \centering
	\begin{longtable}{|p{6cm}|p{8cm}|}
	\hline
		\multicolumn{2}{|l|}{\textbf{UC5: Indstil placering af lampen}} \\\hline
		\endfirsthead
		
		\multicolumn{2}{l}{...fortsat fra forrige side} \\ \hline 
		\multicolumn{2}{|l|}{\textbf{UC5: Indstil placering af lampen}} \\\hline
		\endhead		

        \multicolumn{2}{r}{fortsættes på næste side...} \\
        \endfoot
        \endlastfoot
        
        \textbf{Mål}								
            & Brugeren har indstillet lampens placering i rummet.
        \\ \hline
        \textbf{Initialisering}					
            & Bruger trykker på touchskærmen.
        \\ \hline
        \textbf{Aktører og Stakeholders}			
            & Bruger.
        \\ \hline
        \textbf{Referencer}						
            & Ingen.
        \\ \hline
        \textbf{Antal af samtidige hændelser}	
            & Ingen.
        \\ \hline
        \textbf{Forudsætning}					
            & Systemet er tændt, kalibreret og funktionsdygtigt. Touchskærmen viser 'Position'-fanen.
        \\ \hline
        \textbf{Efterfølgende tilstand}			
            & Lampen hænger i en valgt position.
        \\ \hline
        \textbf{Hovedforløb}						
            &
            \begin{enumerate}
                \item Bruger sætter X, Y og Z sliderne til en valgt position under 'Position'-fanen.
                \item Bruger trykker på Go-knappen.
                \item Lampen flytter sig til den valgte position.
                [Undtagelse 1: Afstandssensor registrerer objekt]
            \end{enumerate}
        \\ \hline
        \textbf{Undtagelser}						
            & [Undtagelse 1: Afstandssensor registrerer objekt]
            \begin{enumerate}
			    \item System stopper Z-motoren.
			    \item Lampen hejses 1 cm op fra dens position.
		    \end{enumerate}
        \\ \hline
	\end{longtable}
	\label{UC6} 
\end{center}