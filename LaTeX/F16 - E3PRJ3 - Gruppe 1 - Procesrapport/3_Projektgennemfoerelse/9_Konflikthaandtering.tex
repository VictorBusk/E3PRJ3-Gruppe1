\section{Konflikthåndtering}

Vores første konflikt opstod forholdsvis sent i processen. Et gruppemedlem var bagefter med sin opgave, og som følge af dette kom andre bagud med deres opgaver, da den første opgaves fuldførelse var en nødvendighed for fortsættelse. 
Problemstillingen blev vendt på et gruppemøde og forslag til løsninger blev drøftet. Opgavens ansvarlige fik til den følgende mandag til at færdiggøre sin opgave, ellers ville et andet gruppemedlem overtage opgaven og færdiggøre denne. Det lykkes opgavens originale ansvarlige at færdiggøre sin opgave til den nye deadline. 
Grundet denne konflikt kom vi en smule bagud med projektet, men vi fik løst det. Det udløste desværre en kædereaktion for projektet, hvor vi kom lidt bagud generelt. Dette gav en lidt trykket stemning. Dette blev løst, da et af gruppens medlemmer skrev et indlæg i gruppens Facebook-gruppe hvor vedkommende ytrede sig om de irritationer og frustrationer vedkommende var påvirket af. Efterfølgende skrev gruppens øvrige medlemmer indlæg, hvor de forholdte sig til disse irritationer og frustrationer, samt gav deres forklaring på grundlaget til eventuel medvirken og skabelse af disse. Dette gav god luft for gruppens moral, og arbejdet kunne genoptages effektivt.