\section{Jeppe}

Jeg påtog mig tidligt i projektet en lederrolle i gruppen da det falder mig naturligt. Vi blev hurtigt, som gruppe, enige om vores mødestruktur, arbejdsmetoder og ansvarsområder, hvilket gjorde at vi tidligt kunne komme igang med projektet. Som gruppe har vi fungeret godt og harmonisk. Vi fik dog hen imod slutningen af projektet udfordringer som gruppe da aftaler og deadlines ikke blev overholdt. Det frustrerede mig meget personligt, da vi ellers havde haft et godt drive i gruppen og nærmede os afslutningen af projektet. Jeg kunne hurtigt se at det ikke vil være muligt at få et færdigt produkt hvis vi ikke fik taget fat om vores udfordringer. Så efter at havde kigget indad valgte jeg at adressere vores problemstillinger som en gruppeenhed, og kom med et forslag til hvordan vi som gruppe kunne komme videre. Det blev der taget godt imod fra gruppens side, og brugt konstruktivt af gruppemedlemmerne.
Fagligt har jeg i projektet primært arbejdet med software, hvilket er et område jeg har et utroligt godt drive inden for. Det er gået rigtig mange timer og et par "næsten" søvnløse nætter, da jeg ikke har kunne slippe det, hvilket har været en positiv oplevelse for mig personligt. Jeg er blevet udfordret da hovedparten af det kode jeg har skrevet har været i C sprog, herunder udvikling af SPI kernemodulet til DevKit8000 og alt koden til PSoC'en. Det har været sjovt at mærke at man har udviklet sig fagligt i forløbet og iterativt har kunne optimere den kode man har fået skrevet.